\newpage
\maketitle
\begin{center}
\Large \textbf{第4章 回测平台} \quad 
\end{center}
\begin{abstract}
在本章中我们将详细讲解基于强化学习环境的回测平台。
\end{abstract}
\section{回测平台概述}
在本章中,我们基于强化学习环境,创建一个策略的回测平台。
\subsection{启动入口}
下面我们来看启动的入口函数:
\lstset{language=PYTHON, caption={回测平台入口}, label={c004-rl-backtest-entry-point}, basicstyle = \ttfamily}
\begin{lstlisting}
def run(self):
    stock_symbol = 'sh600260'
    model = self.build_model()
    env = FmtsEnv(stock_symbol)
    obs = env.reset()        
    done = False 
    action = env.action_space.sample() 
    idx = 0
    while not done:
        X = self.get_model_X(obs)
        quotation_state = self.get_quotation_state(model, X)
        if quotation_state == 0:
            # 买入
            action[0] = 0.5
            action[1] = 1.0
        elif quotation_state == 1:
            # 卖出
            action[0] = 1.5
            action[1] = 1.0
        else:
            # 持有
            action[0] = 2.5
            action[1] = 0.5
        obs, reward, done, info = env.step(action)
        if env.current_step > 200:
            done = True

def build_model(self):
    '''
    强化学习环境Reset中需要调用本函数,初始化模型
    '''
    cmd_args = self.parse_args()
    print('command line args: {0};'.format(cmd_args))
    batch_size = cmd_args.batch_size
    NUM_CLS = 3
    cmd_args.embedding_size = 5
    seq_length = 11
    cmd_args.depth = 6
    cmd_args.num_heads = 8
    model = FmtsTransformer(emb=cmd_args.embedding_size, heads=cmd_args.num_heads, depth=cmd_args.depth, \
                seq_length=seq_length, num_tokens=cmd_args.vocab_size, num_classes=NUM_CLS, \
                max_pool=cmd_args.max_pool)
    model.to(self.device)
    e, model_dict, optimizer_dict = self.load_ckpt(self.ckpt_file)
    model.load_state_dict(model_dict)
    return model
\end{lstlisting}
代码解读如下所示:
\begin{itemize}
    \item 第3行:创建模型,具体代码见第28行;
    \item 第4行:创建强化学习环境,具体见下一节;
    \item 第5行:重置强化学习环境,并获得系统初始观察值;
    \item 第6行:将是否结束回测置为否;
    \item 第7行:从行动空间中随机抽样出一个行动,行动是一个二维数组,第1个元素代表操作,在[0,1]之间为买入,在(1,2]之间为卖出,(2,+$\infty$)时为持有,第二个
    元素为操作的百分比;
    \item 第9行:如果结束回测标志不为真则一直循环;
    \item 第10行:根据环境的观察求出用于模型的输入信号;
    \item 第11行:通过运行模型,确定当前时刻的市场行情状态:0-上涨、1-下跌、2-震荡;
    \item 第12$\sim$15行:当处于上涨行情时,如果还有现金,则执行买入操作,action[0]取小于1的数,action[1]指定使用当前现金的百分比;
    \item 第16$\sim$19行:当处于下跌行情时,如果持有股票,则执行卖出操作,action[0]取1至2之间的数,action[1]指定使用当前持股数的百分比;
    \item 第20$\sim$23行:当处于震荡行情时,action[0]取大于2的数,action[1]的值被忽略;
    \item 第24行:将行动传给环境执行,环境会返回:obs-当前的状态(执行完行动后)、reward-该行动获得的奖励、done-是否结束回测标志、info-其他附加信息;
\end{itemize}

\subsection{强化学习环境}
2021.09.28

\section{总结}