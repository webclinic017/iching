\newpage
\maketitle
\begin{center}
\Large \textbf{第3章 模型训练} \quad 
\end{center}
\begin{abstract}
在本章中我们将详细讲解用于金融交易的Transformer网络的模型训练和预测过程。
\end{abstract}
\section{模型训练与预测概述概述}
\subsection{训练过程}
下面我们来看模型的训练过程,训练入口程序如下所示:
\lstset{language=PYTHON, caption={模型训练入口}, label={c003-train-entry-point}, basicstyle = \ttfamily}
\begin{lstlisting}
    def train(self):
        cmd_args = self.parse_args()
        stock_symbol = 'sh600260'
        batch_size = cmd_args.batch_size
        NUM_CLS = 3
        cmd_args.embedding_size = 5
        seq_length = 11
        cmd_args.num_heads = 4
        cmd_args.depth = 6 # 原始值为2
        train_iter, test_iter = self.load_stock_dataset(stock_symbol, batch_size)
        cmd_args.num_heads = 8
        model = FmtsTransformer(emb=cmd_args.embedding_size, heads=cmd_args.num_heads, depth=cmd_args.depth, \
                    seq_length=seq_length, num_tokens=cmd_args.vocab_size, num_classes=NUM_CLS, \
                    max_pool=cmd_args.max_pool)
        model.to(self.device)
        opt = torch.optim.Adam(lr=cmd_args.lr, params=model.parameters())
        sch = torch.optim.lr_scheduler.LambdaLR(opt, lambda i: min(i / (cmd_args.lr_warmup / cmd_args.batch_size), 1.0))
        if cmd_args.continue_train:
            e, model_dict, optimizer_dict = self.load_ckpt(self.ckpt_file)
            model.load_state_dict(model_dict)
            opt.load_state_dict(optimizer_dict)
        # training loop
        cmd_args.num_epochs = 3
        seen = 0
        # early stopping参数
        best_acc = -1
        acc_up = 0.0
        min_acc_up = 0.000001 # 识别为精度提高的最小阈值
        non_acc_up_epochs = 0 # 目前多少个epoch精度未提高
        max_no_acc_up_epochs = 50 # 如果精度在这些epoch后还没提高则终止训练过程
        for epoch in range(cmd_args.num_epochs):
            print(f'\n epoch {epoch}')
            model.train(True)
            for batch in tqdm.tqdm(train_iter):
                opt.zero_grad()
                X, y = self.get_stock_batch_sample(batch, batch_size, cmd_args.embedding_size)
                y_hat = model(X)
                loss = F.nll_loss(y_hat, y)
                loss.backward()
                # clip gradients
                # - If the total gradient vector has a length > 1, we clip it back down to 1.
                if cmd_args.gradient_clipping > 0.0:
                    nn.utils.clip_grad_norm_(model.parameters(), cmd_args.gradient_clipping)
                opt.step()
                sch.step()
                seen += X.size(0)
            with torch.no_grad():
                model.train(False)
                tot, cor= 0.0, 0.0 
                for batch in tqdm.tqdm(test_iter):
                    X, y = self.get_stock_batch_sample(batch, batch_size, cmd_args.embedding_size)
                    y_hat = model(X).argmax(dim=1)
                    tot += float(X.size(0))
                    cor += float((y == y_hat).sum().item())
                acc = cor / tot
                # 获取当前最佳测试集精度,并保存对应的模型
                if best_acc < acc:
                    acc_up = acc - best_acc
                    if acc_up > min_acc_up:
                        best_acc = acc
                        non_acc_up_epochs = 0
                        print('保存模型参数')
                        self.save_ckpt(self.ckpt_file, epoch, model, opt)
                else:
                    non_acc_up_epochs += 1
                    if non_acc_up_epochs > max_no_acc_up_epochs:
                        print('模型已经处于饱合状态,停止训练过程')
                        break
                print(f'-- {"test" if cmd_args.final else "validation"} accuracy {acc:.3}')
\end{lstlisting}
代码解读如下所示:
\begin{itemize}
    \item 第5行:Transformer网络输出三种市场状态:上涨、下跌、震荡;
    \item 第6行:我们将每个时刻等价为一个单词,每个时刻可以用开盘、最高、最低、收盘、交易量来表示,因此相当于每个单词是5维向量;
    \item 第7行:我们通常考虑当前时刻,同时向前看10个时刻,因此每个样本包括11个时刻,相当于每个句子有11个单词,所以序列长度为11;
    \item 第8行:共有6层结构;
    \item 第10行:共有8个自注意头;
    \item 第12$\sim$15行:初始化模型,其中vocab\_size=50000为缺省值,max\_pool代表使用最大池化,并将其放入GPU中;
    \item 第16行:使用adam优化算法;
    \item 第17行:使用带有warmup的学习调整计划算法:...;
    \item 第18$\sim$21行:在之前训练的基础上继续训练;
    \item 第22行:训练集训练遍数;
    \item 第23行:共训练了多少个样本数;
    \item 第26行:当前所取得的最佳精度;
    \item 第27行:当前精度提高的数值;
    \item 第行:;
    \item 第行:;
    \item 第行:;
    \item 第行:;
    \item 第行:;
    \item 第行:;
    \item 第行:;
    \item 第行:;
    \item 第行:;
    \item 第行:;
\end{itemize}

\subsection{预测过程}

\section{总结}