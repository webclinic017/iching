\newpage
\maketitle
\begin{center}
\Large \textbf{第1章 行情数据处理} \quad 
\end{center}
\begin{abstract}
在本章中我们将通过AKshare库,获取A股分钟级行情数据,并将其进行预处理,变为深度学习可用的数据集。
\end{abstract}
\section{行情数据处理概述}
\subsection{获取原始行情数据}
我们首先通过apps.fmts.ds.akshare\_data\_source.AkshareDataSource获取原始的行情数据,-将其保存到csv文件中。如果存在该csv文件,则
直接从该文件中读出数据并返回。数据格式为:
\lstset{language=PYTHON, caption={行情数据格式}, label={c001-quotation-1-minute-bar}, basicstyle = \ttfamily}
\begin{lstlisting}
......
['2021-08-17 14:55:00', 30.339999999999996, 30.339999999999996, 30.339999999999996, 30.339999999999996, 200.0]
['2021-08-17 14:55:01', 30.339999999999996, 30.339999999999996, 30.339999999999996, 30.339999999999996, 200.0]
......
\end{lstlisting}
\subsection{行情数据预处理}
\subsubsection{价格折线图}
我们以收盘价为例,收盘价的折线图绘制程序如下所示:
\lstset{language=PYTHON, caption={收盘价折线图}, label={c001-close-price-curve-001}, basicstyle = \ttfamily}
\begin{lstlisting}
    class OhlcvProcessor(object):
    # 价格折线图模式
    PCM_DATETIME = 1
    PCM_TICK = 2

    @staticmethod
    def draw_close_price_curve(stock_symbol: str, mode=1) -> None:
        '''
        绘制收盘价折线图,横轴为时间,纵轴为收盘价
        '''
        data = AkshareDataSource.get_minute_bars(stock_symbol=stock_symbol)
        x = [v[0] for v in data[0:1000]]
        y = [v[4] for v in data[0:1000]]
        if mode == OhlcvProcessor.PCM_DATETIME:
            OhlcvProcessor._draw_date_price_curve(x, y)
        else:
            OhlcvProcessor._draw_tick_price_curve(y)

    def _draw_date_price_curve(x: List, y: List) -> None:
        x = [datetime.datetime.strptime(di, '%Y-%m-%d %H:%M:%S') for di in x]
        fig, axes = plt.subplots(1, 1, figsize=(8, 4))
        plt.rcParams['font.sans-serif']=['SimHei'] #用来正常显示中文标签
        plt.rcParams['axes.unicode_minus'] = False #用来正常显示负号
        # 最大化绘图窗口
        figmanager = plt.get_current_fig_manager()
        figmanager.window.state('zoomed')    #最大化
        # 绘制收盘价格折线图
        axes.plot_date(x, np.array(y), '-', label='Net Worth')
        # 设置横轴时间显示格式
        axes.xaxis.set_major_formatter(DateFormatter('%Y-%m-%d %H:%M:%S'))
        plt.gcf().autofmt_xdate()
        # 显示图像
        plt.show()
    
    def _draw_tick_price_curve(y: List) -> None:
        x = range(len(y))
        fig, axes = plt.subplots(1, 1, figsize=(8, 4))
        plt.rcParams['font.sans-serif']=['SimHei'] #用来正常显示中文标签
        plt.rcParams['axes.unicode_minus'] = False #用来正常显示负号
        # 最大化绘图窗口
        figmanager = plt.get_current_fig_manager()
        figmanager.window.state('zoomed')    #最大化
        # 绘制收盘价格折线图
        plt.title('收盘价折线图')
        axes.set_xlabel('时间刻度')
        axes.set_ylabel('收盘价')
        axes.plot(x, np.array(y), '-', label='Net Worth')
        plt.show()
\end{lstlisting}
代码解读如下所示:
\begin{itemize}
    \item 第3、4行:定义收盘价曲线绘制方式,一种是横轴为时间,另一种横轴为行情序号;
    \item 第6$\sim$10行:定义收盘价绘制方法,参数为股票代码和绘制模式,缺省值为横轴为时间(以分钟为单位),这种模式的缺点是从上一日收盘到下一日开盘有
    较大的时间间隔;
    \item 第11行:获取分钟线行情数据,格式为:[[dateteime, open, high, low, close, volume]];
    \item 第19行:以横轴为行情时间值绘制收盘价曲线;
    \begin{itemize}
        \item 第20行:将时间变为'2021-08-21 12:56:00'格式的列表;
        \item 第21行:设置显示图形;
        \item 第22行:设置字体使matplotlib可以正确显示汉字;
        \item 第23行:使matplotlib可以显示负号;
        \item 第24$\sim$26行:使matplotlib绘图窗口最大化;
        \item 第27、28行:绘制收盘价时间曲线;
        \item 第29$\sim$31行:设置横坐标轴时间显示格式为'2021-08-21 12:56:00',并自动调整为45度角倾斜,以节省显示空间;
    \end{itemize}
\end{itemize}
\begin{figure}[H]
	\caption{以时间为横轴的收盘价折线图}
	\label{f000001}
	\centering
	\includegraphics[width=10cm]{images/f000001}
\end{figure}
\begin{figure}[H]
	\caption{以序号为横轴的收盘价折线图}
	\label{f000002}
	\centering
	\includegraphics[width=10cm]{images/f000002}
\end{figure}
如\ref{f000001}所示,图中每天收盘到第二天开盘间没有行情数据,所以图形不太好看出规律,而图\ref{f000002}则可以较好的反映价格的变化规律,因此
我们在通常情况下,选择图\ref{f000002}的形式。
\subsubsection{对数差分序列}
我们都知道,原始的行情数据,不具备平稳性,即无法通过历史数据来预测未来,而对数差分序列则具有平稳性,可以用来进行预测。如下所示:
\lstset{language=PYTHON, caption={收盘价折线图}, label={c001-close-price-curve-001}, basicstyle = \ttfamily}
\begin{lstlisting}
    @staticmethod
    def gen_1d_log_diff_norm(stock_symbol, items):
        '''
        从原始行情数据,求出一阶对数收益率log(day2)-log(day1),然后求出每列均值和标准差,利用
        (x-mu)/std进行标准化,分别保存原始信息和归整后信息
        参数:
            stock_symbol 股票编号
            items 由AkshareDataSource.get_minute_bars方法获取到
        '''
        datas = np.array([x[1:] for x in items])
        log_ds = np.log(datas)
        log_diff = np.diff(log_ds, n=1, axis=0)
        log_diff_mu = np.mean(log_diff, axis=0)
        log_diff_std = np.std(log_diff, axis=0)
        ld_ds = (log_diff - log_diff_mu) / log_diff_std
        # 保存原始信息
        raw_file = './apps/fmts/data/{0}_1m_raw.txt'.format(stock_symbol)
        with open(raw_file, 'w', encoding='utf-8') as fd:
            for item in items[1:]:
                fd.write('{0},{1},{2},{3},{4},{5}\n'.format(item[0], item[1], 
                            item[2], item[3], item[4], item[5]))
        # 保存规整化后数据
        ld_file = './apps/fmts/data/{0}_1m_ld.csv'.format(stock_symbol)
        np.savetxt(ld_file, ld_ds)

# 测试程序
    def test_gen_1d_log_diff_norm_001(self):
        stock_symbol = 'sh600260'
        items = AkshareDataSource.get_minute_bars(stock_symbol=stock_symbol)
        OhlcvProcessor.gen_1d_log_diff_norm(stock_symbol, items)
\end{lstlisting}
代码解读如下所示:
\begin{itemize}
    \item 第2行:items由AkshareDataSource.get\_minute\_bars方法获取到,格式为[..., ['2021-08-19 15:00:00', 1.1, 1.5, 1.0, 1.2, 1000], ...];
    \item 第10行:把items中的条目,去除掉日期列后,生成ndarray;
    \item 第11行:对所有元素取对数,以自然数e为底,np.log2是以2底,np.log10是以10为底;
    \item 第12行:取一阶差分,其中n=1代表是一阶差分,即后面一个元素减前面一个元素,out[i] = x[i+1]-x[i],因为axis=0,所以i代表行;
    \item 第13行:求出行方向的均值;
    \item 第14行:求出行方向的标准差;
    \item 第15行:进行归一化:$\hat{x}=\frac{x-\mu}{\sigma}$;
    \item 第18~21行:保存原始的行情信息,因为取了一阶差分,所以去掉了第1行;
    \item 第23、24行:保存一阶差分规整化后的数据;
\end{itemize}
\subsection{数据集支撑数据}
在每一个时间点,我们向前看window\_size个时间点,缺省是10个,然后再加上当前时间点的数据:开盘、最高、最低、收盘、交易量,所以共有$10 \times 5 + 5 = 55$个数据,
我们的算法会根据这55维向量,我们以当前时刻收盘价为标准,确定判断为上涨趋势的最低价格(一旦超过该值即视为上涨),判断为下跌的最高价格(一旦低于该值即视为下跌),向后
连续读取指定个时刻,缺省值为100,如果未来价格首先高于上涨趋势的最低价格,则将当前时刻判断为上涨状态,如果我们有资金,就应该进行买入操作;如果未来价格首先低于下跌趋
势的最高价格,则将当前时刻判断为下跌状态,此时如果我们有持仓,则应卖出持有的股票,如果既没高于上涨趋势的最低价格,也没低于下跌趋势的最高价格,则将当前时刻判断为震荡
状态,此时不进行任何操作。我们先来看数据的生成:
\lstset{language=PYTHON, caption={获取数据集后面原始数据}, label={c001-get-row-dataset-data}, basicstyle = \ttfamily}
\begin{lstlisting}
    @staticmethod
    def get_ds_raw_data(stock_symbol: str, window_size: int=10, forward_size: int=100) -> Tuple[np.ndarray, np.ndarray, List[str]]:
        '''
        获取数据集所需数据
        stock_symbol 股票代码
        window_size 从当前时间点向前看多少个时间点
        forward_size 向后看多少个时间点确定市场行情是上涨、下跌和震荡
        返回值 
            X 连续11个时间点的OHLCV的数据,形状为n*55,一阶Log差分形式
            y 某个时间点及其前10个时间点行情数据组成的shapelet对应的行情(按Box方式确定):0-震荡;1-上升;2-下跌;
            info 当前时间刻行情的真实值
        '''
        print('获取数据集数据')
        # 获取行情数据
        quotations = OhlcvProcessor.get_quotations(stock_symbol)
        # 获取归整化行情数据
        log_1d_datas = []
        log_1d_file = './apps/fmts/data/{0}_1m_ld.csv'.format(stock_symbol)
        with open(log_1d_file, 'r', encoding='utf-8') as fd:
            for row in fd:
                row = row.strip()
                arrs = row.split(' ')
                item = [arrs[0], arrs[1], arrs[2], arrs[3], arrs[4]]
                log_1d_datas.append(item)
        # 
        ldd_size = len(log_1d_datas) - forward_size
        print('ldd_size: {0};'.format(ldd_size))
        X_raw = []
        for pos in range(window_size, ldd_size, 1):
            item = []
            for idx in range(pos-window_size, pos):
                item += log_1d_datas[idx]
            item += log_1d_datas[pos]
            X_raw.append(item)
        X = np.array(X_raw, dtype=np.float32)
        ds_X_csv = './apps/fmts/data/{0}_1m_X.csv'.format(stock_symbol)
        np.savetxt(ds_X_csv, X, delimiter=',')
        # 获取行情状态
        y = np.zeros((X.shape[0],), dtype=np.int64)
        OhlcvProcessor.get_market_state(y, quotations, window_size, forward_size)
        # 获取日期和真实行情数值
        raw_datas = []
        raw_data_file = './apps/fmts/data/{0}_1m_raw.txt'.format(stock_symbol)
        seq = 0
        with open(raw_data_file, 'r', encoding='utf-8') as fd:
            for row in fd:
                if seq >= window_size and seq<ldd_size:
                    row = row.strip()
                    arrs = row.split(',')
                    item = [arrs[0], float(arrs[1]), float(arrs[2]), float(arrs[3]), float(arrs[4]), float(arrs[5])]
                    raw_datas.append(item)
                seq += 1
        a1 = len(raw_datas)
        return X[:a1], y[:a1], raw_datas
\end{lstlisting}
代码解读如下所示:

\section{最后}